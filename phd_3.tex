% !TEX TS-program = pdflatex
% !TEX encoding = UTF-8 Unicode

% This is a simple template for a LaTeX document using the "article" class.
% See "book", "report", "letter" for other types of document.

\documentclass[12pt]{article} % use larger type; default would be 10pt

\usepackage[utf8]{inputenc} % set input encoding (not needed with XeLaTeX)
\usepackage{booktabs}
\newcommand{\ra}[1]{\renewcommand{\arraystretch}{#1}}

%%% Examples of Article customizations
% These packages are optional, depending whether you want the features they provide.
% See the LaTeX Companion or other references for full information.

%%% PAGE DIMENSIONS
\usepackage[margin=2.5 cm]{geometry}
\usepackage{blindtext} % to change the page dimensions
\geometry{a4paper} % or letterpaper (US) or a5paper or....
% \geometry{margin=2in} % for example, change the margins to 2 inches all round
% \geometry{landscape} % set up the page for landscape
%   read geometry.pdf for detailed page layout information

\usepackage{graphicx} % support the \includegraphics command and options
\usepackage{epstopdf}
% \usepackage[parfill]{parskip} % Activate to begin paragraphs with an empty line rather than an indent
\usepackage{pgfplots}
\pgfplotsset{compat=1.13}
\usepackage{booktabs,caption,fixltx2e}
\usepackage[flushleft]{threeparttable}
\usepackage{color, colortbl}
\definecolor{Gray}{gray}{0.9}
%%% PACKAGES
\usepackage{placeins}
\usepackage{booktabs} % for much better looking tables
\usepackage{array} % for better arrays (eg matrices) in maths
\usepackage{paralist} % very flexible & customisable lists (eg. enumerate/itemize, etc.)
\usepackage{verbatim} % adds environment for commenting out blocks of text & for better verbatim
\usepackage{subfig} % make it possible to include more than one captioned figure/table in a single float
% These packages are all incorporated in the memoir class to one degree or another...
\usepackage{amsmath}
\usepackage{cases}
\usepackage{graphicx}
\usepackage{float}
\usepackage{authblk}
\usepackage{pgfplots}
\usepackage{pdfpages}
\linespread{1.5}


%%% The "real" document content comes below...
\title{Research Project- PhD in Economics}
\author{Fabio Blasutto}

\begin{document}
	\maketitle
	

\section{The rise of the Divorce gradient and}

Starting from the 1960 until 1980 the divorce rate in the U.S. has increased drastically from 2 to more than 6 divorces per 1000 population per year. Then, after 1980, the divorce rate started diminishing again, but the pattern has been completely different for college graduate and the unskilled population: as Martin (2004) observed, in fact; while the divorce rate has been diminishing for the first group, it has slightly increased for the second group. It can be also observed that educational assortative mating is rising and that the age at first marriage is increasing and it is higher for college graduates (Greenwood et al. (2016)). So far it has not been developed a theory that can explain all these facts all together. In particular, it is not clear why the trend in divorces has been diverging for the two different educational groups.

Recently in the US and in most developed countries an increase of divorce rates has been observed. The main reason of this change can be attributed to the decline of usual gains from marriage, such as specialization, due to technological changes. Now, the marriage rate is lower, divorces have increased and a new form has emerged: cohabitation. This new for of relationship thought is not used by every strand of the population/ in fact it is particularly common among the non-skilled. The divorce rate is also particularly strong for the non skilled. This means that some gains from marriage are still present: in particular, marriage can be seen either as a commitment technology or a signal of a good match. Why do we observe such differences? 
\\
\begin{thebibliography}{widestlabel}
	
\bibitem{Martin}\textsc{Martin, S. P.} (2004). Growing evidence for a “divorce divide”? Education and marital dissolution rates in the US since the 1970s. \textit{Series on social dimensions of inequality}. New York: Russel Sage Foundation Working Papers.
\bibitem{Grenwood} \textsc{Greenwood, J., Guner, N., Kocharkov, G., \& Santos}, C. (2016). Technology and the changing family: A unified model of marriage, divorce, educational attainment, and married female labor-force participation. \textit{American Economic Journal: Macroeconomics}, 8(1), 1-41.
\end{thebibliography}
\end{document}	